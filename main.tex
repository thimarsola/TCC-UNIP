\documentclass[
	% -- opções da classe memoir --
	12pt,				% tamanho da fonte
	openright,			% capítulos começam em pág ímpar (insere página vazia caso preciso)
	oneside,			% para impressão apenas em frente. Oposto a twoside
	a4paper,			% tamanho do papel. 
	% -- opções da classe abntex2 --
	chapter=TITLE,		% títulos de capítulos convertidos em letras maiúsculas
	%section=TITLE,		% títulos de seções convertidos em letras maiúsculas
	%subsection=TITLE,	% títulos de subseções convertidos em letras maiúsculas
	%subsubsection=TITLE,% títulos de subsubseções convertidos em letras maiúsculas
	% -- opções do pacote babel --
	english,			% idioma adicional para hifenização
	brazil				% o último idioma é o principal do documento
	]{abntex2}
	

\usepackage{silence}
\WarningFilter{latex}{You have requested package}

\pdfstringdefDisableCommands{\let\uppercase\relax}

\usepackage{longtable}

\usepackage[final]{pdfpages} % Incluir pdf


\usepackage{float} %para imagens
% Para mais formas de enumerar itens
\renewcommand{\labelenumi}{\theenumi}

% Para indicar copyright e registered
\usepackage{textcomp}

% Para imagens e legendas
\usepackage[footnotesize]{caption}
\usepackage{subcaption}

% Para editar forma das equações
\usepackage{amsmath}

\usepackage{siunitx}
\usepackage{amsmath}
\usepackage{amssymb}
\usepackage{amsfonts}
\DeclareMathOperator{\sinc}{sinc}
\DeclareMathOperator{\vecE}{vec}
\newtheorem{mydef}{Defini{\c{c}}\~ao}


% ---
% Pacotes básicos 
% ---	
\usepackage{helvet} % Usa o equivalente ARIAL
\renewcommand{\familydefault}{\sfdefault}
\usepackage[T1]{fontenc}		% Selecao de codigos de fonte.
\usepackage[utf8]{inputenc}		% Codificacao do documento (conversão automática dos acentos)
\usepackage{lastpage}			% Usado pela Ficha catalográfica
\usepackage{indentfirst}		% Indenta o primeiro parágrafo de cada seção.
\usepackage{color}				% Controle das cores
\usepackage{graphicx}			% Inclusão de gráficos
\usepackage{microtype} 			% para melhorias de justificação
% ---
		
% ---
% Pacotes adicionais, usados apenas no âmbito do Modelo Canônico do abnteX2
% ---
\usepackage{lipsum}				% para geração de dummy text
% ---


% ---
% Pacotes de citações
% ---

%\usepackage[brazilian,hyperpageref]{backref}	 % Paginas com as citações na bibl


\usepackage[alf]{abntex2cite}	% Citações padrão ABNT


\usepackage{tikz}
\usetikzlibrary{shapes,arrows,positioning,calc}
\usepackage{tikz-3dplot}


% --- 
% CONFIGURAÇÕES DE PACOTES
% --- 

% ---
% Configurações do pacote backref
% Usado sem a opção hyperpageref de backref

%\renewcommand{\backrefpagesname}{Citado na(s) página(s):~}


% Texto padrão antes do número das páginas


%\renewcommand{\backref}{}
% Define os textos da citação
%\renewcommand*{\backrefalt}[4]{
%	\ifcase #1 %
%		Nenhuma citação no texto.%
%	\or
%		Citado na página #2.%
%	\else
%		Citado #1 vezes nas páginas #2.%
%	\fi}%
% ---

% ---
% Informações de dados para CAPA e FOLHA DE ROSTO
% ---


\autor{
	\uppercase{
		Brenon Costa Ortega\\
		Darlan Gomes de Oliveira Filho\\
		Kauê Brescancini Martin\\
		Renan Hideki hirashiki\\
		Renan Pereira da Silva\\
		Thiago Dias de Jesus Marsola Corrêa
	}
}
\titulo{COLOQUE SEU TITULO}
\local{São Paulo}
\data{2020}
\orientador{Prof. MsC. Wanys Arnaldo Antonio Rocha}
\coorientador{Prof. MsC. Romeu Fontana Júnior}
\coorientador{Prof. MsC. Ademir Antonio dos Santos}
\preambulo{Trabalho de Conclusão de Curso apresentado ao curso de Bacharelado em Engenharia de Controle e Automação da Universidade Paulista -- Unip, como pré-requisito para obtenção do grau de bacharel em Engenharia de Controle e Automação.}
% ---

\usepackage{listings}
\renewcommand\lstlistlistingname{Lista de códigos}
\renewcommand\lstlistingname{Código}
\newlistof{lstlistoflistings}{lol}{\lstlistlistingname}

\definecolor{mygreen}{RGB}{28,172,0} % color values Red, Green, Blue
\definecolor{mylilas}{RGB}{170,55,241}

% configuracoes do listings

\lstset{language=Matlab,%
    basicstyle=\tiny,
    breaklines=true,%
    morekeywords={matlab2tikz},
    keywordstyle=\color{blue},%
    morekeywords=[2]{1}, keywordstyle=[2]{\color{black}},
    identifierstyle=\color{black},%
    stringstyle=\color{mylilas},
    commentstyle=\color{mygreen},%
    showstringspaces=false,%without this there will be a symbol in the places where there is a space
    numbers=left,%
    numberstyle={\tiny \color{black}},% size of the numbers
    numbersep=9pt, % this defines how far the numbers are from the text
    emph=[1]{for,end,break},emphstyle=[1]\color{red}, %some words to emphasise
    %emph=[2]{word1,word2}, emphstyle=[2]{style}, 
    numberbychapter=false,
    linewidth=\textwidth,
    xleftmargin=0.05\textwidth,
    xrightmargin=0.05\textwidth,
    frame=single,
}


% %% TESTE PARA LISTA DE QUADROS
% % Novo list of (listings) para QUADROS

% \newcommand{\quadroname}{Quadro}
% \newcommand{\listofquadrosname}{Lista de quadros}

% \newfloat[chapter]{quadro}{loq}{\quadroname}
% \newlistof{listofquadros}{loq}{\listofquadrosname}
% \newlistentry{quadro}{loq}{0}

% % configurações para atender às regras da ABNT
% \setfloatadjustment{quadro}{\centering}
% \counterwithout{quadro}{chapter}
% \renewcommand{\cftquadroname}{\quadroname\space} 
% \renewcommand*{\cftquadroaftersnum}{\hfill--\hfill}

% % Configuração de posicionamento padrão:
% \setfloatlocations{quadro}{hbtp}



%\listingsfontinline
%\def\listingsfont{\ttfamily}


% ---
% Configurações de aparência do PDF final

% alterando o aspecto da cor azul
\definecolor{blue}{RGB}{0,0,0}


% informações do PDF
\makeatletter
\hypersetup{
     	%pagebackref=true,
		pdftitle={\@title}, 
		pdfauthor={\@author},
    	pdfsubject={\imprimirpreambulo},
	    pdfcreator={LaTeX with abnTeX2},
		pdfkeywords={abnt}{latex}{abntex}{abntex2}{trabalho acadêmico}, 
		colorlinks=true,       		% false: boxed links; true: colored links
    	linkcolor=blue,          	% color of internal links
    	citecolor=blue,        		% color of links to bibliography
    	filecolor=magenta,      		% color of file links
		urlcolor=blue,
		bookmarksdepth=4
}
\makeatother
% --- 

% --- 
% Espaçamentos entre linhas e parágrafos 
% --- 
% tamanho da assinatura
\setlength{\ABNTEXsignwidth}{10cm}

% O tamanho do parágrafo é dado por:
\setlength{\parindent}{1.3cm}

% Controle do espaçamento entre um parágrafo e outro:
\setlength{\parskip}{0.2cm}  % tente também \onelineskip

% ---
% compila o indice
% ---
\makeindex
% ---

% ----
% Alterando a lista de Códigos
% ----

\AtBeginDocument{% the counter is defined later
	\counterwithout{lstlisting}{chapter}%
}
\makeatletter
\renewcommand{\l@lstlisting}[2]{%
	\@dottedtocline{1}{0em}{1.5em}{\lstlistingname\ #1}{#2}%
}
\makeatother






% ----
% Início do documento
% ----
%\numberwithin{figure}{chapter} %numeração com o capitulo
%\numberwithin{table}{chapter}%numeração com o capitulo
\begin{document}

% Seleciona o idioma do documento (conforme pacotes do babel)
%\selectlanguage{english}
\selectlanguage{brazil}

% Retira espaço extra obsoleto entre as frases.
\frenchspacing 

% ----------------------------------------------------------
% ELEMENTOS PRÉ-TEXTUAIS
% ----------------------------------------------------------
% \pretextual

% ---
% Capa
% ---
\imprimircapa
% ---

% ---
% Folha de rosto
% (o * indica que haverá a ficha bibliográfica)
% ---
\imprimirfolhaderosto
% ---

% ---
% Inserir a ficha bibliografica
% ---

% Isto é um exemplo de Ficha Catalográfica, ou ``Dados internacionais de
% catalogação-na-publicação''. Você pode utilizar este modelo como referência. 
% Porém, provavelmente a biblioteca da sua universidade lhe fornecerá um PDF
% com a ficha catalográfica definitiva após a defesa do trabalho. Quando estiver
% com o documento, salve-o como PDF no diretório do seu projeto e substitua todo
% o conteúdo de implementação deste arquivo pelo comando abaixo:
%





\begin{fichacatalografica}
 	\sffamily
 	\vspace*{\fill}					% Posição vertical
 	\begin{center}					% Minipage Centralizado
 	\fbox{\begin{minipage}[c][8cm]{13.5cm}		% Largura
 	\small
 	\imprimirautor
 	%Sobrenome, Nome do autor

  	\hspace{0.5cm} \imprimirtitulo  / \imprimirautor. --
  	\imprimirlocal, \imprimirdata-

  	\hspace{0.5cm} \pageref{LastPage} p. : il. (algumas color.) ; 30 cm.\\

  	\hspace{0.5cm} \imprimirorientadorRotulo~\imprimirorientador\\
	
  	\hspace{0.5cm}
  	\parbox[t]{\textwidth}{\imprimirtipotrabalho~--~\imprimirinstituicao,
  	\imprimirdata.}\\
	
  	\hspace{0.5cm}
  		1. Média Tensão.
  		2. \textit{Power Line Communication}
  		2. Simulação.
  		I. Prof. Dr. Omar Alexander Chura Vilcanqui.
  		II. Universidade Federal do Acre.
  		III. Centro de Ciências Exatas e Tecnológicas.
  		IV. Envio de Informação pela rede elétrica: estudo de caso do alimentador da Universidade Federal do Acre.		
  	\end{minipage}}
  	\end{center}
\end{fichacatalografica}
% ---


% ---
% Inserir folha de aprovação
% ---

% Isto é um exemplo de Folha de aprovação, elemento obrigatório da NBR
% 14724/2011 (seção 4.2.1.3). Você pode utilizar este modelo até a aprovação
% do trabalho. Após isso, substitua todo o conteúdo deste arquivo por uma
% imagem da página assinada pela banca com o comando abaixo:
%
% \includepdf{folhadeaprovacao_final.pdf}
%
\begin{folhadeaprovacao}

  \begin{center}
    {\ABNTEXchapterfont\bfseries\large COMISSÃO JULGADORA}

    \vspace*{\fill}\vspace*{\fill}
    \begin{center}
      {\ABNTEXchapterfont\bfseries\large Trabalho de Conclusão de Curso}
    \end{center}
    \vspace*{\fill}      
    \vspace*{\fill}
   \end{center}
        
   \noindent Autores:\\ \imprimirautor\\
   \\Data da Defesa: 15/07/2019\\
   Título do Trabalho: \imprimirtitulo
   
   \vspace*{\fill}      
   \vspace*{\fill}

   \assinatura{\textbf{\imprimirorientador} \\ Orientador -- CCET/UNIP} 
   \assinatura{\textbf{Prof. MsC. Romeu Fontana Júnior} \\ Co-orientador -- CCET/UNIP}
   \assinatura{\textbf{Prof. MsC. Ademir Antonio dos Santos} \\ Co-orientador CCET/UNIP}
     
   \vspace*{\fill}      
   \vspace*{\fill}  
     
\end{folhadeaprovacao}
% ---


% Agradecimentos
% ---
\begin{agradecimentos}
Escrever aqui os agradecimentos....
\end{agradecimentos}
% ---

% ---
% Epígrafe
% ---
\begin{epigrafe}
    \vspace*{\fill}
		\begin{flushright}
			\textit{''A ciência é uma grande montanha de açúcar; dessa montanha só conseguimos retirar 			insignificantes pedacinhos''\\
			(Malba Tahan)}
	\end{flushright}
\end{epigrafe}
% ---

% ---
% RESUMOS
% ---

% resumo em português
\setlength{\absparsep}{18pt} % ajusta o espaçamento dos parágrafos do resumo
\begin{resumo}

 \textbf{Palavras-chave}: Modelagem matemática, controlador em avanço de fase, controle preditivo generalizado, conversor buck.
\end{resumo}

% resumo em inglês
\begin{resumo}[Abstract]
 \begin{otherlanguage*}{english}

   \vspace{\onelineskip}
 
   \noindent 
   \textbf{Keywords}: Mathematical modeling, phase advance controller, generalized predictive control, buck converter.
 \end{otherlanguage*}
\end{resumo}

% ---
% inserir lista de ilustrações
% ---
\pdfbookmark[0]{\listfigurename}{lof}
\listoffigures*
\cleardoublepage
% ---

% ---
% inserir lista de tabelas
% ---
\pdfbookmark[0]{\listtablename}{lot}
\listoftables*
\cleardoublepage
% ---

% ---
% inserir lista de códigos
% ---

\pdfbookmark[0]{\lstlistlistingname}{lol}
\lstlistoflistings*
\cleardoublepage
% ---

% % ---
% % inserir lista de quadros
% % ---
% \pdfbookmark[0]{\listofquadrosname}{loq}
% \listofquadros*
% \cleardoublepage
% % ---

% ---
% inserir lista de abreviaturas e siglas
% ---
%\begin{siglas}
%\end{siglas}

% ---
% inserir lista de símbolos
% ---
%\begin{simbolos}
%  \item[$v_k$] Tensão instantânea no condutor k.
%  \item[$V_k$]
%\end{simbolos}
% ---

% ---
% inserir o sumario
% ---
\pdfbookmark[0]{\contentsname}{toc}
\tableofcontents*
 \cleardoublepage
% ---


% ----------------------------------------------------------
% ----------------------------------------------------------
% ELEMENTOS TEXTUAIS
% ----------------------------------------------------------
% ----------------------------------------------------------
\textual

% ----------------------------------------------------------
% Introdução
% ----------------------------------------------------------
\chapter[Introdução]{Introdução\label{cap:introducao}}
%\addcontentsline{toc}{chapter}{Introdução}
% ----------------------------------------------------------





% ---

% ----------------------------------------------------------
% Cadeira de Rodas
% ----------------------------------------------------------
\chapter[Cadeira de Rodas]{Cadeira de Rodas\label{cap:cadeiraRodas}}
%\addcontentsline{toc}{chapter}{Introdução}
% ----------------------------------------------------------

\section{História}
\section{Acessibilidade}
\section{Mobilidade urbana para deficientes}
\section{Modelos e custos}
% ---

% ----------------------------------------------------------
% Projeto de Engenharia
% ----------------------------------------------------------
\chapter[Projeto de Engenharia]{Projeto de Engenharia\label{cap:projeto}}
Neste capítulo é apresentado conceitos e conhecimentos sobre projetos de engenharia. Serão desenvolvidos conhecimentos sobre a metodologias aplicadas, normas, ferramentas computacionais, análise de falhas, prototipagem e tecnologias de materiais e manufatura.

\section{Conceito de Projeto}
Cada vez mais as organizações e as pessoas utilizam o conceito de projeto para alcançar os seus objetivos e/ou metas, normalmente estabelecidos através de planejamento organizacional ou pessoal.
\par
De acordo com Norton (\citeyear{nort13}) ``O termo projeto(\textit{``design''}) claramente engloba uma grande variedade de significados''.
\par
Segundo \uppercase{pmbok\textregistered} (\citeyear{pmbok13}) ``Projeto é um esforço temporário empreendido para criar um produto, serviço ou resultado exclusivo. A natureza temporária dos projetos indica que eles têm um início e um término definidos.''.



\section{Normas}
\section{Ferramentas Computacionais}
\section{FMEA}
\section{Prototipagem}
\section{Materiais}
\section{Tecnologias de Manufatura}
% ---

% ----------------------------------------------------------
% Controle
% ----------------------------------------------------------

% ---

% ----------------------------------------------------------
% Resultados e Discussões
% ----------------------------------------------------------

% ---

% ----------------------------------------------------------



% Finaliza a parte no bookmark do PDF
% para que se inicie o bookmark na raiz
% e adiciona espaço de parte no Sumário
% ----------------------------------------------------------
\phantompart

% ----------------------------------------------------------
% Conclusões
% ----------------------------------------------------------
% ---
% Conclusão
% ---
\chapter{Conclusão}



\section{Dificuldades encontradas}




%Um último ponto possível para estudo, seria aprofundar o uso da tecnologia PLC em aplicações concretas de envio de informação pela rede de MT, por exemplo, para a detecção de faltas. Este estudo pode estar associado a uma comparação financeira, tendo em vista avaliar o custo benefício de ter um sistema como este pronto para uso em determinadas empresas ou órgãos públicos.

% ---

% ----------------------------------------------------------
% ELEMENTOS PÓS-TEXTUAIS
% ----------------------------------------------------------
\postextual
% ----------------------------------------------------------

% ----------------------------------------------------------
% Referências bibliográficas
% ----------------------------------------------------------
\bibliography{Referencias}


% ----------------------------------------------------------
% Glossário
% ----------------------------------------------------------
%
% Consulte o manual da classe abntex2 para orientações sobre o glossário.
%
%\glossary

% ----------------------------------------------------------
% Apêndices
% ----------------------------------------------------------

% ---
% Inicia os apêndices
% ---
% \begin{apendicesenv}

% % Imprime uma página indicando o início dos apêndices
% \partapendices

% % ----------------------------------------------------------
% \chapter{Nullam elementum urna vel imperdiet sodales elit ipsum pharetra ligula
% ac pretium ante justo a nulla curabitur tristique arcu eu metus}
% % ----------------------------------------------------------
% \lipsum[55-57]

% \end{apendicesenv}
% % ---


% ----------------------------------------------------------
% Anexos
% ----------------------------------------------------------

% ---
% Inicia os anexos
% ---
%\begin{anexosenv}
%\end{anexosenv}

% %---------------------------------------------------------------------
% % INDICE REMISSIVO
% %---------------------------------------------------------------------
% \phantompart
% \printindex
% %---------------------------------------------------------------------

\end{document}
